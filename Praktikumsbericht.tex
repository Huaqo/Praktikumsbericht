\documentclass[12pt,a4paper]{report}
\usepackage[utf8]{inputenc}
\usepackage[german]{babel}
\usepackage{setspace}
\usepackage{hyperref}

\title{Praktikumsbericht}
\author{ \\ Matrikelnummer:  \\ Studienfach: }
\date{\today}

\newcommand{\companyinfo}{
    \begin{center}
        \textbf{} \\
        Abteilung:  \\
        Einsatzbereich:  \\
        Zeitpunkt und Dauer des Praktikums:  \\
        Betreuer:  \\
        Ort: Berlin
    \end{center}
}

\onehalfspacing % Zeilenabstand von 1,5

\begin{document}
\begin{titlepage}
    \begin{center}
        \vspace*{1cm}
        
        \textbf{\Large Praktikumsbericht}
        
        \vspace{1.5cm}
        
        \textbf{Joaquin Gottlebe} \\
        Matrikelnummer: 5429130 \\
        Studienfach: Geographische Wissenschaften
        
        \vspace{1.5cm}
        
        \textbf{Freie Universität Berlin} \\
        Abteilung: Institut für Geographische Wissenschaften \\
        Einsatzbereich: Fernerkundung und Geoinformatik \\
        Zeitpunkt und Dauer des Praktikums: 01.07.2021 - \today \\
        Betreuer: Prof. Dr. Fabian Fassnacht \\
        Ort: Berlin
        
        \vspace{1.5cm}
        
        \textbf{\today}
    \end{center}
\end{titlepage}

\chapter{Einleitung}

\onehalfspacing
Das Praktikum, das ich absolviert habe, fand an der Freien Universität Berlin
im Institut für Geowissenschaften, speziell in dem Bereich der Fernerkundung
und Geoinformatik, statt. Das Institut befindet sich in Lankwitz im Süden von
Berlin. Hier befinden sich auch andere Bereiche wie die Physische- und
Humangeographie. Der Campus bietet neben den Einrichtungen auch große
Grünflächen zum Erholen, eine Bibliothek und eine Mensa. Die Abteilung
Fernerkundung und Geoinformatik selbst hat 6 feste Mitarbeiter, ergänzt durch
mehrere Wissenschaftler, Doktoranden und studentische Hilfskräfte. Alle
arbeiten gemeinsam daran, Fernerkundungstechniken und GIS-Methoden
weiterzuentwickeln, um methodische und ökologische Fragestellungen zu
beantworten. Ein besonderer Fokus liegt auf dem Verständnis von anthropogenen
Landnutzungs- und Landbedeckungsänderungen, insbesondere in Bezug auf Wald- und
Graslandökosysteme.
%Aufgaben der Abteilung}
Die primären Verantwortlichkeiten der Abteilung liegen in den Bereichen
Forschung und Lehre. 
%
In der Forschung wird mit modernsten Technologien und Methoden gearbeitet, um
beispielsweise Fernerkundungsdaten zu analysieren oder GIS-basierte Modelle zu
erstellen. Erdbeobachtungssysteme und fortschrittliche
Machine-Learning-Techniken kommen zum Einsatz, um räumlich verteilte und
zeitlich häufige Daten zu liefern und zu analysieren.
%
Das derzeitige Projekt FutureForest (future-forest.eu) zielt darauf ab,
künstliche Intelligenz (KI) einzusetzen, um Informationen bereitzustellen, die
eine klimaangepasste Waldbewirtschaftung unterstützen können. Die Hauptaufgaben
der FU Berlin in diesem Kontext umfassen die Sentinel-2-Zeitreihenerkennung von
Waldschäden in Deutschland und das operationelle 'Echtzeit'-Monitoring mittels
Deep Learning, die frühzeitige Erkennung beschädigter Gebiete und die
Bereitstellung einer deutschlandweiten Karte aktueller Waldschäden. 
%
Auf der Lehrseite bietet die Abteilung eine Vielzahl von Modulen an, die den
Studierenden eine breite und tiefe Kenntnis der Geowissenschaften vermitteln.
Sowohl im Bachelor als auch im Master werden Module angeboten.  Es gibt sowohl
grundlegende Kurse wie Geographische Informationssysteme als auch
spezialisierte Module wie Fernerkundung und Geoinformatik. Diese Kurse ergänzen
die Forschungsaktivitäten und ermöglichen es den Studierenden, ihre
theoretischen und praktischen Kenntnisse in realen Projekten anzuwenden.

\chapter{Hauptteil}

%Berufsfeld und Aufgaben
In meiner Rolle als studentische Hilfskraft hatte ich die Gelegenheit, ein
breites Spektrum an Tätigkeiten zu erleben. Dies reichte von der pädagogischen
Unterstützung in Form von Tutoring bis hin zur aktiven Beteiligung an
Forschungsprojekten. Durch diese Erfahrungen konnte ich feststellen, wie
vielfältig und bereichernd die Aufgaben in der akademischen Welt sein können,
sowohl für meine fachliche als auch für meine persönliche Entwicklung.
%
Diese Vielfalt spiegelte sich auch in der Funktion des wissenschaftlichen
Mitarbeiters wider, die ich aus nächster Nähe beobachten konnte. Diese Rolle
ist äußerst facettenreich und bietet die Möglichkeit, intensiv an
Forschungsprojekten teilzunehmen, die akademische Lehre zu gestalten und
Forschungsergebnisse in wissenschaftlichen Fachzeitschriften zu
publizieren.
%Bewerbungsverlauf
Meine Kontaktaufnahme mit dem Institut war eher traditionell. Ich stieß auf
eine Ausschreibung für die Praktikumsstelle auf der Website der Universität,
bewarb mich darauf und wurde zu einem digitalen Bewerbungsgespräch eingeladen,
da wir uns mitten in der Corona-Pandemie befanden. Das Gespräch selbst war
überraschend locker und persönlich; es wurde von einem ehemaligen Professor, PD
Dr. Sören Hese, und einer wissenschaftlichen Mitarbeiterin, Dr. Marion
Stellmes, durchgeführt. Sie stellten nicht nur fachliche, sondern auch viele
persönliche Fragen, um mich besser kennen zu lernen. Dies gab mir ein positives
Gefühl und machte mich zuversichtlich, dass sie sich für mich entscheiden
würden.
%
Es gab keine speziellen Tests oder Aufgaben im Auswahlprozess, nur einige
Fragen zum Verständnis, die ich beantworten musste. Nur kurze Zeit nach dem
Interview erhielt ich eine E-Mail, die mir die Zusage für die Praktikumsstelle
übermittelte. Ich war sehr erfreut über diese Nachricht und konnte es kaum
erwarten, mit den Arbeiten zu beginnen.
%
Obwohl es keine formale Einführung oder Schulung gab, waren die anfänglichen
Aufgaben einfach und überschaubar, was mir den Einstieg erleichterte. Mit der
Zeit haben sich die Aufgaben jedoch weiterentwickelt und sind komplexer
geworden, was den gesamten Prozess umso spannender und lehrreicher
gestaltete.
%Erwartungen und Ziele
Vor Beginn des Praktikums hatte ich mehrere Erwartungen und Ziele. Zunächst
wollte ich tiefer in das Arbeitsleben von Wissenschaftlern eintauchen. Fragen
zum Alltag, zu den spezifischen Aufgaben und zur Feldarbeit bewegten mich. Ich
war neugierig, ob Feldarbeit anstrengend oder spaßig ist und ob mir die vielen
Stunden vor dem Computer zu schaffen machen würden.
%
Ein weiterer wichtiger Einflussfaktor war mein Vorbild Richard Feynman, dessen
Ansätze sowohl als Wissenschaftler als auch als Mentor ich sehr bewundere. Ich
erhoffte mir, dass das Praktikum mir ermöglichen würde, ähnliche Qualitäten und
Arbeitsweisen kennenzulernen.
%
Mein idealer Tag während des Praktikums wäre gefüllt mit einer Vielzahl von
Aufgaben und erreichbaren Zielen, die mir eine breite Palette von Erfahrungen
bieten würden. Darüber hinaus sah ich das Praktikum als Gelegenheit, meine an
der Universität erworbenen Kenntnisse in GIS und Fernerkundung zu vertiefen. Da
diese Bereiche sehr eng mit meinem Studienfach verknüpft sind, war ich
begeistert von der Möglichkeit, nicht nur theoretische, sondern auch praktische
Einblicke in echte Arbeitsabläufe zu erhalten.
%
Zusätzlich zu den fachlichen Aspekten erhoffte ich mir auch, meine Soft Skills
wie Teamwork, Kommunikation und Arbeitsmanagement zu verbessern. Diese
Fähigkeiten sind in der wissenschaftlichen Arbeit ebenso wichtig wie die
Fachkenntnisse selbst, und ich sah das Praktikum als perfekte Gelegenheit, sie
zu entwickeln und zu verfeinern.
%Tätigkeitsfelder
Während meines Praktikums an der Freien Universität Berlin im Institut für
Geowissenschaften hatte ich die Gelegenheit, in einer Vielzahl von
anspruchsvollen und vielseitigen Tätigkeiten involviert zu sein. Die
Anforderungen, die an mich gestellt wurden, kamen sowohl von den Mitarbeitern
als auch von Prof. Dr. Fabian Fassnacht und waren sehr vielfältig. Sie
umfassten fachliche und organisatorische Aspekte und forderten von mir sowohl
Eigeninitiative als auch zuverlässiges Arbeiten. Ich empfand es als besonders
erfüllend, wenn ich spürte, dass ich einen wirklichen Beitrag zur Forschung und
Lehre leisten konnte.
%
Meine Hauptaufgaben umfassten das Tutoring für Studierende, wissenschaftliche
Zuarbeiten für Dozenten und die Mitarbeit in der Feldarbeit. In der Lehre
assistierte ich bei verschiedenen Kursen und Seminaren, die sich mit Themen wie
GIS und Fernerkundung beschäftigten. Dabei war ich eng in die Vorbereitung von
Unterrichtsmaterialien involviert und arbeitete bei der Erstellung von
Präsentationen und Skripten eng mit den Dozenten zusammen.
%
Ein weiterer wichtiger Aspekt meines Praktikums war die Mitbetreuung der
Sprechstunden für Studierende, bei denen ich als Ansprechpartner für ihre
Fragen und Anliegen fungierte. Darüber hinaus war ich in die Vorbereitung und
Durchführung von FU-internen Veranstaltungen eingebunden, die von der Planung
bis zur logistischen Umsetzung reichten.
%
In der Forschung hatte ich die Gelegenheit, bei Geländearbeiten mit modernsten
Technologien wie Laserscannern, Drohnen und Spektrometern mitzuwirken. Diese
praktischen Erfahrungen ermöglichten es mir, meine im Studium erworbenen
Kenntnisse direkt anzuwenden und mein technisches Know-how zu erweitern.
%
Last but not least konnte ich meine Fähigkeiten im Bereich der Koordination,
des Managements und der Kommunikation durch die enge Zusammenarbeit mit dem
Team und den Teamleitern vertiefen. Diese Soft Skills erwiesen sich als ebenso
wichtig für meine persönliche und berufliche Entwicklung wie die fachlichen
Kenntnisse, die ich während des Praktikums erwerben konnte.
%Einsatz von Fachkenntnissen
Ich hatte die Gelegenheit, meine während des Studiums erworbenen Fachkenntnisse
in vielfältig einzusetzen. Insbesondere im Bereich von GIS und Fernerkundung
konnte ich meine Kenntnisse nicht nur in der Forschung für die Datenauswertung
nutzen, sondern auch in der Lehre, wo ich die Module betreute, in denen ich
selbst ausgebildet wurde. Dabei half ich Studierenden, sich in diesen
Fachgebieten zurechtzufinden.
%
Meine Programmierkenntnisse mit R kamen ebenfalls zum Einsatz. Ich war
beispielsweise für das Auslesen und die Vorverarbeitung von Daten eines
Spektrometers verantwortlich, was eine sorgfältige und präzise Arbeitsweise
erforderte. Zudem erwiesen sich meine allgemeinen EDV-Kenntnisse und meine
Fertigkeiten im Umgang mit MS-Office als äußerst nützlich. Die Organisation der
Lehre wurde vorwiegend über Word und Excel koordiniert, was meine tägliche
Arbeit effizienter machte.
%
Meine eigenständige und sorgfältige Arbeitsweise ermöglichte es mir, die mir
übertragenen Aufgaben effizient und zuverlässig zu erfüllen. Mein
geographisches Grundwissen kam vor allem bei der Kartenerstellung und bei der
Erstellung von Plots zum Tragen. Es half mir nicht nur bei der
Dateninterpretation, sondern auch beim tieferen Verständnis der
Forschungsergebnisse und deren Kontextualisierung.
%
Darüber hinaus waren meine Englischkenntnisse von großem Vorteil, insbesondere
beim Verstehen und Verfassen wissenschaftlicher Texte. Dies war besonders
nützlich, da einige Forschungsprojekte und Publikationen in englischer Sprache
verfasst waren. Zudem musste ich in vielen Fällen auf Englisch kommunizieren,
was meine internationalen Arbeitsbeziehungen stärkte.
%Arbeitsalltag
Die Arbeitszeiten variierten je nach den aktuellen Anforderungen und Projekten.
Ein typischer Tag begann mit einer Fahrradfahrt zur Universität, gefolgt von
Kaffee und der Erstellung einer To-Do-Liste. Nach dem Mittagessen mit Kollegen
setzte ich meine Arbeit fort, um schließlich den Feierabend einzuläuten und
nach Hause zu fahren. Besonders stressige Zeiten waren der Beginn des
Semesters, wenn die Lehre vorbereitet werden musste, und die wöchentlichen
Abgabetermine für Hausaufgaben, die ich innerhalb eines Tages bewerten
musste.
%Betreuung und Teamintegration
Die Betreuung während des Praktikums war exzellent. Ich wurde Teil eines Teams
aus sechs festen Mitarbeitern, ergänzt durch weitere Wissenschaftler,
Doktoranden und studentische Hilfskräfte. Die Teamdynamik war geprägt durch
flache Hierarchien und eine lockere, konstruktive Arbeitsatmosphäre.
Wöchentliche Team-Meetings trugen zur Teamintegration bei und halfen mir, mich
schnell in die jeweiligen Projekte einzufinden.
%Schwierigkeiten und Lösungsansätze
Zu Beginn des Praktikums gab es einige Herausforderungen bei der Einarbeitung
in neue Arbeitsabläufe und Forschungsgeräte. Besonders hilfreich war hierbei
die Unterstützung meiner Kollegen, die mich sorgfältig einarbeiteten und mir
viel zutrauten. Ein prägendes Erlebnis war, als mir eine Kollegin vor ihrem
Urlaub ein Spektrometer für eigenständige Messungen anvertraute. Diese
Herausforderungen bewältigte ich durch systematisches Arbeiten und die aktive
Suche nach Lösungen.
%Erworbene Kenntnisse und Fähigkeiten
Das Praktikum bot mir die Möglichkeit, eine Reihe von wichtigen Kenntnissen und
Fähigkeiten zu erwerben oder zu vertiefen. Insbesondere konnte ich meine
Fähigkeiten im wissenschaftlichen Arbeiten und im Umgang mit Studierenden
verbessern. Besonders erinnerungswert war ein Erlebnis mit einem Studenten, bei
dem ich durch gezieltes Nachfragen dazu beitrug, dass er eine Lösung für sein
Problem selbst fand. Darüber hinaus gewann ich praktische Erfahrungen im Umgang
mit speziellen Werkzeugen und Technologien wie Laserscannern, Spektrometern und
Drohnen.

\chapter{Fazit}

Das Praktikum an der Freien Universität Berlin im Institut für
Geowissenschaften war für mich eine äußerst lehrreiche und motivierende
Erfahrung. Es hat meine Erwartungen weitgehend erfüllt und mir wertvolle
Einblicke in die wissenschaftliche Arbeit und Forschung gegeben.
%Einfluss auf Studienverlauf und Berufswahl
Das Praktikum hat meine Entschlossenheit gestärkt, meinen Studienweg konsequent
fortzusetzen und mich weiter in den Gebieten der Geowissenschaften und
Fernerkundung zu spezialisieren. Insbesondere hat mich der Bereich der
anthropogenen Landnutzungs- und Landbedeckungsänderungen und deren Auswirkungen
auf spezifische Ökosysteme fasziniert. Prof. Dr. Fabian Fassnacht hat dabei
einen besonders prägenden Einfluss auf meine Berufswahl und Studienrichtung
gehabt. Durch die gewonnenen Erfahrungen habe ich viele neue Ideen für mein
weiteres Studium gesammelt und bin nun auch offen für die Möglichkeit einer
akademischen Laufbahn, vielleicht sogar bis zur Promotion.
%Perspektiven und Möglichkeiten
Die positiven Erfahrungen und das gute Arbeitsklima haben meine Perspektive auf
die akademische Welt und die Möglichkeiten innerhalb dieser erweitert. Nach dem
Praktikum bin ich in die Arbeitsgruppe Mensch-Umwelt Modellierung gewechselt,
mit der wir bereits zuvor zusammengearbeitet hatten. Dies hat mein Netzwerk
erweitert, und ich bin zuversichtlich, dass ich meine Kollegen in der Zukunft
für gemeinsame Projekte kontaktieren kann.
%Bewertung des Praktikums
Insgesamt bewerte ich das Praktikum sehr positiv. Besonders hervorzuheben sind
die spannenden Themen und die gute Teamdynamik, die maßgeblich zu dieser
positiven Erfahrung beigetragen haben. Verbesserungswürdig sind eventuell die
Kommunikation und die Integration neuer Mitglieder ins Team, obwohl dies
bereits auf einem guten Niveau ist.
%Empfehlung für andere Studierende
Ich würde anderen Studierenden definitiv empfehlen, ein Praktikum in diesem
Institut zu absolvieren. Die vielfältigen Tätigkeitsbereiche, das angenehme
Arbeitsklima und die hervorragende Betreuung bieten eine optimale Grundlage für
persönliche und fachliche Entwicklung. Die wichtigste Lektion, die ich mitnehme
und anderen mitgeben möchte, ist: Einfach bewerben und ins kalte Wasser
springen, auch wenn man Angst hat. Man wird so viel lernen und
weiterkommen.

\end{document}
